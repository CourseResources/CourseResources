\documentclass[12pt]{article}

\usepackage{amsmath}
\usepackage{amssymb}
\usepackage{amsthm}
\usepackage{hyperref}
\usepackage[pdftex]{graphicx}
%\usepackage{physymb}
%\usepackage{wrapfig}
\usepackage{braket}
\usepackage{subcaption}
\title{PH 304: Assignment 3}

\author{Manish Goregaokar (120260006)}
\date{February 15, 2015}
\begin{document}
\maketitle

\section*{Problem 1}
Calculating the regular way,

The initial entropy is ($\frac32 k_B T=E_1/N_1=E_2/N_2$)

\begin{align*}
S_i = S_1 + S_2 &= N_1k_B\left(\ln V_1\left(\frac{4\pi em}{3}\frac E N\right)^{3/2}\right) + N_2 \dots\\
&= N_1k_B(\ln V_1 + s) + N_2k_B(\ln V_2 + s) \qquad s = \ln \left(\frac{4\pi em}{3}\frac E N\right)^{3/2}
\end{align*}

The final entropy is 

\begin{align*}
S_f  &=  (N_1+N_2)k_B\ln\left((V_1+V_2)\left(\frac{4\pi em}{3}\frac{E_1+E_2}{N_1+N_2}\right)^{3/2}\right)\\
 &=  (N_1+N_2)k_B\ln\left((V_1+V_2)\left(\frac{4\pi em}{3}\frac E N\right)^{3/2}\right)\\
 &=N_1 k_B\ln(V_1 +V_2) + N_2 k_B \ln (V_1 + V_2) + k_B(N_1 +N_2)s
\end{align*}


Subtracting, we get that the change in entropy is $\Delta S = N_1k_B\ln\frac{V}{V_1}+ N_2k_B\ln\frac{V}{V_2}$

However, this has overcounted situations where effectively identical molecules are distributed across the barrier. We must remove all cases of exchanges, which involves dividing the original $\Omega$ by $N!$.

This manifests itself as a decrease in entropy by $k_B\ln N!$, which gives a net difference of $k_B\ln N!-k_B\ln N_1!-k_B\ln N_2!$

\begin{align*}
\Delta S_{corr} &= k_B(\ln N!-\ln N_1!-\ln N_2!)\\
&= k_B ((N_1+N_2)\ln (N_1+N_2) - N_1\ln N_1 -N_2\ln N_2)\\
&= k_B (N_1\ln\frac N N_1 +N_2\ln\frac N N_2 )\\
&= k_B (N_1\ln\frac V V_1 +N_2\ln\frac V V_2 )\qquad\because \frac N V = \frac{N_1} V_1 = \frac{N_2} V_2\\
&= \Delta S
\end{align*}

Since the necessary correction $\Delta S_{corr}$ is the same as the original change in entropy, The net change, $\Delta S - \Delta S_{corr} = 0$, so the net entropy of mixing is zero.

\section*{Problem 2}

The total number of states of this system is $\Omega(E,N)$. If a single oscillator has energy $\epsilon$, the number of states with this energy will be the number of states of $N-1$ oscillators of energy $E-\epsilon$ since they are independent. Thus, the probability is $P(\epsilon)=\frac{\Omega(E-\epsilon,N-1)}{\Omega(E,N)}$

Since $\Omega(E,N)=\frac{(2\pi)^N}{N!}\left(\frac{2E}{\omega}\right)^N$, we get that this probability distribution is $\frac{N! \left(\frac{E}{\omega }\right){}^{-N} \left(\frac{E-e}{\omega }\right){}^{N-1}}{4 \pi  (N-1)!}$, which simplifies to $\frac{N \omega  E^{-N} \left(E-e\right){}^{N-1}}{4 \pi }$

Thus, the probability distribution is $P(e) = \frac{N \omega  E^{-N} \left(E-e\right){}^{N-1}}{4 \pi }$, where $e = \frac{p^2}{2 m}+\frac{1}{2} m q^2 \omega ^2$. At large $N$, we get $P(\epsilon) = \frac{\omega  e^{-\frac{\epsilon}{T k_B}}}{4 \pi  T k_B}$ since $E=Nk_B T$, thus getting $\boxed{P(p,q) = \frac{\omega  e^{-\frac{\frac{p^2}{2 m}+\frac{1}{2} m q^2 \omega ^2}{T k_B}}}{4 \pi  T k_B}}$

Integrating $\int_{-\infty}^\infty\int_{-\infty}^\infty\frac{p^2}{2m} P(p,q)\mathrm dp\mathrm dq$, we get $\boxed{\frac{k_B T}{4}}$, the average kinetic energy of the individual oscillator. Integrating  $\int_{-\infty}^\infty\int_{-\infty}^\infty\frac{m\omega q^2}{2} P(p,q)\mathrm dp\mathrm dq$, we also get $\boxed{\frac{k_B T}{4}}$, which is the average potential energy of the individual oscillator.
\section*{Problem 3}

We need to choose $n$ atoms out of $N$ to remove them, so the number of ways to remove $n$ atoms is $\boxed{^NC_n}$

We need to choose an ordered set of $n$ sites out of $M$ , so we have $\boxed{^MC_n}$ as the number of ways to place these $n$ atoms.

Thus, the total number of equiprobable states is $\Omega(N,M,n) = ^NC_n\times^MC_n$
\begin{align*}
\therefore S(E, N, M) &= k_B\ln \Omega\\
&= k_B\ln \left(\frac{N!}{n!(N-n)!}\frac{M!}{n!(M-n)!}\right)\\
&=k_B(N\ln N -N + M\ln M - M -n\ln n + n -(N-n)\ln(N-n)\\ &\quad+ N - n -(M-n)\ln(M-n) + M - n -n\ln n + n)\\
&= k_B(N\ln N + M\ln M  -(N-n)\ln(N-n)-(M-n)\ln(M-n)-2 n\ln n)\\
&= k_B(N\ln N + M\ln M  -(N-n)\ln(N-n)-(M-n)\ln(M-n)\\&\quad\quad-n\ln\frac{n}{N} -n\ln\frac{n}{N}-n\ln N - n\ln M)\\
&= -k_B \left(n\ln\frac n N + n\ln\frac n M + (N-n)\ln\left(1- \frac n N\right)+ (M-n)\ln\left(1- \frac n M\right) \right)
\end{align*}

Thus, entropy is $\boxed{-k_B \left(n\ln\frac n N + n\ln\frac n M + (N-n)\ln\left(1- \frac n N\right)+ (M-n)\ln\left(1- \frac n M\right) \right)}$, where $n=\frac E\Delta$


Now, $\frac 1 T = \frac{\partial S}{\partial E}= \frac{\partial S}{\Delta\partial n}$

\begin{align*}
\therefore \frac{\Delta}{k_B T}& = \frac{\partial S}{\partial n}\\
&=  \log \left(1-\frac{n}{N}\right)-\log \left(\frac{n}{N}\right))+\log \left(1-\frac{n}{M}\right)-\log \left(\frac{n}{M}\right)\\
&= \ln\left(\frac{N}{n} -1\right)+\ln\left(\frac{M}{n} -1\right)\\
&= \ln\left(\frac{N\Delta}{E}-1\right)+ \ln\left(\frac{N\Delta}{E}-1\right)
\end{align*}

Thus, $\boxed{T =\frac{\Delta}{k_B\ln\left(\frac{N\Delta}{E}-1\right)+ \ln\left(\frac{N\Delta}{E}-1\right)}}$

Rearranging, we have $\ln\left(\frac{N}{n} -1\right)+\ln\left(\frac{M}{n} -1\right) = \frac{\Delta}{k_B T}$, or $\ln\frac{(N-n)(M-n)}{n^2}=\frac{\Delta}{k_B T}$, so we get $\boxed{\frac{n^2}{(N-n)(M-n)} = e^{-\frac{\Delta}{k_B T}}}$

For large $\Delta$, $\frac{n^2}{(N-n)(M-n)} = 0$, so $\boxed{n = 0}$. For small $\Delta$, $\frac{n^2}{(N-n)(M-n)} = 1$, we get $\boxed{n = \frac{MN}{M+N}}$


If $M=n$, then $e^{\frac{\Delta}{k_B T}} = \frac{(N-n)^2}{n^2} = (f^{-1}-1)^2$, where $f=n/N$

This gives us $f =\frac{1}{1 + e^{\frac{\Delta}{2k_B T}}}$

For $\Delta=1$, the value of this expression is $\boxed{1.6\times 10^{-17}}$ for $T=300K$, and $\boxed{9.15 \times 10^{-6}}$ for $T=1000K$


\end{document}
