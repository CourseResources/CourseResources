% LaTeX file for a 1 page document
\documentclass[12pt]{article}
\usepackage{amsmath}
\usepackage{amssymb}
\usepackage{amsthm}
\usepackage{physymb}
\usepackage{braket}
\title{EP 307 Assignment 2}

\author{Manish Goregaokar\\120260006}

\begin{document}
\maketitle

\section*{Problem 1}

\begin{align*}
\Braket{e_1|e_1}=\Braket{e_2|e_2} &= 2\\
 \Braket{e_3|e_3} &=4\\
 \Braket{e_1|e_2} &=i\sqrt{2}\\
 \Braket{e_1|e_3} &=1+i\\
 \Braket{e_2|e_3} &=2
\end{align*}

First, we normalize the states to get:

\begin{align*}
\Braket{a_1|a_1}=\Braket{a_2|a_2} &= 1\\
 \Braket{e_3|a_3} &=1\\
 \Braket{a_1|a_2} &=i\frac1{\sqrt{2}}\\
 \Braket{a_1|a_3} &=\frac{1+i}{2\sqrt{2}}\\
 \Braket{a_2|a_3} &=\frac1{\sqrt{2}}
\end{align*}

\newcommand{\proj}[1]{\hat{p}_{a_{#1}}}
Taking $\proj i = \Ket{a_i}\Bra{a_i}$ to be the projection operator with respect to $a_i$, by Gram-Schmidt orthogonalization procedure,

\begin{align*}
&\Ket{\phi_1}=\Ket{a_1}\\
&\Ket{\phi_2}=\Ket{a_2}-\proj1\Ket{a_2}\\
&\Ket{\phi_3}=\Ket{a_3}-\proj1\Ket{a_3}-\proj2\Ket{a_3}\\
\end{align*}
\renewcommand{\proj}[2]{\Ket{a_{#1}}\Braket{a_{#1}|#2}}
Thus
\begin{align*}
&\Ket{\phi_1}=\Ket{a_1}\\
&\Ket{\phi_2}=\Ket{a_2}-\proj1{a_2}\\
&\Ket{\phi_3}=\Ket{a_3}-\proj1{a_3}-\proj2{a_3}\\
\end{align*}
and
\begin{align*}
&\Ket{\phi_1}=\Ket{a_1}\\
&\Ket{\phi_2}=\Ket{a_2}+\frac{1}{\sqrt{2}i}\Ket{a_1}\\
&\Ket{\phi_3}=\Ket{a_3}-\frac{1+i}{2\sqrt{2}}\Ket{a_1}-\frac{1}{\sqrt{2}}\Ket{a_2}\\
\end{align*}

We can calculate \begin{align*}&\Braket{\phi_1|\phi_1}=1\\
&\Braket{\phi_2|\phi_2}=1-\frac{1}{2}=\frac12\\ &\Braket{\phi_3|\phi_3}=1+\frac{(1-i)(1+i)}{8}+\frac12=\frac74
\end{align*}
and from here we get an orthonormal basis

\begin{align*}
&\Ket{\psi_1}=\Ket{a_1}\\
&\Ket{\psi_2}=\sqrt{2}\left(\Ket{a_2}+\frac{1}{\sqrt{2}i}\Ket{a_1}\right)\\
&\Ket{\psi_3}=\frac{2}{\sqrt{7}}\left(\Ket{a_3}-\frac{1+i}{2\sqrt{2}}\Ket{a_1}-\frac{1}{\sqrt{2}}\Ket{a_2}\right)\\
\end{align*}

Rewriting in terms of the original vectors

\begin{align*}
&\Ket{\psi_1}=\frac{1}{\sqrt{2}}\Ket{e_1}\\
&\Ket{\psi_2}=\Ket{e_2}+\frac{1}{\sqrt{2}i}\Ket{e_1}\\
&\Ket{\psi_3}=\frac{4}{\sqrt{7}}\left(\Ket{e_3}-\frac{1+i}{2\sqrt{2}}\Ket{e_1}-\frac{1}{\sqrt{2}}\Ket{e_2}\right)\\
\end{align*}

$\Ket{\psi_1},\Ket{\psi_2},\Ket{\psi_3}$ are our orthonormal basis vectors.

\section*{Problem 2}
First, we normalize them by taking integrals in $[-1,1]$, and get $c_0=\frac1{\sqrt{2}},c_1=\sqrt{\frac32},c_2=\sqrt{\frac52}$

Applying the orthogonalization process,

\begin{align*}
\Ket{\phi_1}&=\Ket{\psi_1}\\
\Ket{\phi_2}&=\Ket{\psi_2}-\Ket{\psi_1}\Braket{\psi_1|\psi_2}\\
\Ket{\phi_3}&=\Ket{\psi_3}-\Ket{\psi_1}\Braket{\psi_1|\psi_3}-\Ket{\psi_2}\Braket{\psi_2|\psi_3}
\end{align*}

calculating the inner products, we get

\begin{align*}
\Ket{\phi_1}&=\Ket{\psi_1}\\
\Ket{\phi_2}&=\Ket{\psi_2}-0\Ket{\psi_1}\\
\Ket{\phi_3}&=\Ket{\psi_3}-\frac{\sqrt5}{2}\frac23\Ket{\psi_1}-0\Ket{\psi_2}
\end{align*}

which evaluates to

\begin{align*}
\phi_1&=\frac1{\sqrt{2}}\\
\phi_2&=\sqrt{\frac32}x\\
\phi_3&=\sqrt{\frac52}\left(x^2-\frac13\right)
\end{align*}
\section*{Problem 3}

$$\hat Bg(x)=g(-x)$$
The operator is linear, since $\hat B(\alpha f(x)+\beta g(x))=\alpha f(-x)+\beta g(-x)=\alpha \hat Bf(x)+\beta\hat B g(x)$

It also is Hermitian, as \begin{align*}\Bra{f}(\hat B\Ket{g})&=\int_{-\infty}^\infty \bar{f}(x)\hat B g(x) dx\\
&= \int_{-\infty}^\infty \bar{f}(x) g(-x) dx\\
&= \int_{-\infty}^\infty \bar{f}(-x)g(x)dx\\
&= \int_{-\infty}^\infty \bar{f}(-x)g(x)dx\\
&=\int_{-\infty}^\infty \overline{\hat B f(x)}g(x)\\
&=\Braket{\overline{\hat B} f|g}\\
\therefore \Braket{f|\hat B g}&=\Braket{\overline{\hat B} f|g}
\end{align*}

To find eigenvalues, $\hat B g(x)=b g(x)=g(-x)$

Since $b g(x)=g(-x)$ and $b g(-x)=g(x)$, $b=\pm 1$ (neglecting the trivial $b=0$ solution).

Thus, we have eigenvectors:
\begin{align*}
&\sum_n a_n x^{2n} & \text{(even function)}\qquad &\text{for eigenvalue $b=1$}\\
&\sum_n b_n x^{2n+1}& \text{(odd function)}\qquad &\text{for eigenvalue $b=-1$}
\end{align*} 
To calculate the commutator,
\begin{align*}
 [B,\hat{x}^n]f&=\hat B \hat x^n f(x)-\hat xn \hat B f(x)\\
 &=\hat B x^n f(x) - \hat x^n f(-x)\\
 &= (-x)^{n}f(-x)-x^n f(-x)\\
 &= f(-x)((-x)^{n}-x^n)
\end{align*}

For this to be zero for all states $f$, $n$ must be even. Thus the commutator is zero only for even $n$
\section*{Problem 4}

$$\hat \Omega = \Ket{\psi}\Bra{\phi}$$

For it to be Hermitian, $\hat \Omega^\dagger =\Omega$. Applying it to the state $\Ket{\Psi}=\Ket{\psi}+\Ket{\phi}$,

\begin{align*}
\Ket{\psi}\Braket{\phi|\Psi}&=\Ket{\phi}\Braket{\psi|\Psi}\\
\Ket{\psi}\Braket{\phi|\psi}+\Ket{\phi}\Braket{\psi|\psi} &= \Ket{\psi}\Braket{\phi|\phi}+\Ket{\phi}\Braket{\psi|\phi}\\
\left(\Braket{\phi|\psi}-\Braket{\phi|\phi}\right)\Ket{\psi} &= \left(\Braket{\psi|\phi}-\Braket{\psi|\psi}\right)\Ket{\phi}
\end{align*}

For this to hold, either $\Ket{\psi}=c\ket{\phi}$ or  $\left(\Braket{\phi|\psi}-\Braket{\phi|\phi}\right)=0=\left(\Braket{\psi|\phi}-\Braket{\psi|\psi}\right)$

In the latter case, we get $\Braket{\psi|\psi}=\Braket{\phi|\psi}=\Braket{\phi|\phi}$, which can only happen if the two states are equivalent.

Thus condition for the operator to be hermitian is that $\Ket{\psi}=c\Ket{\phi}$.

For it to be a projection operator, $c=\frac1{\Braket{\phi|\phi}}$

\section{Problem 5}

$$\hat B \psi(x)=\int_{-\infty}^x x' \psi(x')dx'$$

The eigenvalue problem $\hat B \psi =\lambda\psi$ thus becomes $\int_{-\infty}^x x' \psi(x')dx' = \lambda \psi(x)$.

Differentiating with respect to $x$, this is \begin{align*}
x\psi(x)&=\lambda\psi'(x)\\
\implies \frac{\psi'(x)}{\psi(x)}&=\frac{x}{\lambda}\\
\implies \int_{\psi(0)}^\psi\frac{\psi'(x)}{\psi(x)}&=\int_0^x\frac{x}{\lambda}\\
\implies \log\frac{\psi(x)}{\psi(0)} &= \frac1{\lambda} \frac{x^2}2\\
\implies \psi(x)=\psi(0)e^{\frac{x^2}{2\lambda}}
\end{align*}

This is an acceptable (square-integrable) wavefunction only when $\lambda$ is negative.
\end{document}

\documentclass[12pt]{}