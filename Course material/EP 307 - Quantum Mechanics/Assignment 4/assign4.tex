% LaTeX file for a 1 page document
\documentclass[12pt]{article}
\usepackage{amsmath}
\usepackage{amssymb}
\usepackage{amsthm}
\usepackage{physymb}
\usepackage{braket}
\title{EP 307 Assignment 4}

\author{Manish Goregaokar\\120260006}

\begin{document}
\maketitle

\section*{Problem 1}
$$\prod_{i\neq s}\frac{\hat A-a_i}{a_s-a_i}$$

If $\Ket{a}=\sum c_i\Ket{a_i}$, applying this operator we have \begin{align*}
\prod_{i\neq s}\frac{\hat A-a_i}{a_s-a_i}\left(\sum c_i\Ket{a_i}\right)&=\sum_j\left(\prod_{i\neq s}\frac{\hat A-a_i}{a_s-a_i}\right)c_j\Ket{a_j}\\
&=\left(\prod_{i\neq s}\frac{\hat A-a_i}{a_s-a_i}\right)c_s\Ket{a_s}\\&\qquad\text{Since the other terms have numerator cancelling out for $i=j$}\\
&=\left(\prod_{i\neq s}\frac{a_s-a_i}{a_s-a_i}\right)c_s\Ket{a_s}\\
&=c_s\Ket{a_s}
\end{align*}
Thus the operator is $\frac{\Ket{a_s}\Bra{a_s}}{\Braket{a_s|a_s}}=p_s$ (projection operator)
\section*{Problem 2}
\newcommand{\Tr}{\mathrm{Tr}}
To prove:
$$\Tr(XY)=\Tr(YX)$$

The trace of an operator can be denoted by $\sum_i \Braket{i|A|i}$, where $\Bra{i},\Ket{i}$ are basis vectors corresponding to the $i$th element being $1$ and the rest $0$. 

Thus,

\begin{align*}
\Tr(AB) 
&= \sum_i \Braket{i|AB|i}\\
&= \sum_i  \Braket{i|AIB|i}\\
&= \sum_i\sum_j  \Braket{i|A\Ket{j}\Bra{j}B|i}\\
&= \sum_i\sum_j  \Braket{j|A\Ket{i}\Bra{i}B|j}\\
&= \sum_j  \Braket{j|BIA|j}\\
&= \Tr(BA)
\end{align*}

\section*{Problem 7}

Applying $\hat{A}=\hat{Q}\hat{C}+\hat{C}\hat{Q}$ to an eigenstate $\Ket{\psi_q}$, we get:

\begin{align*}
\hat{A}\Ket{\psi_q} &=(\hat{Q}\hat{C}+\hat{C}\hat{Q})\Ket{\psi_q}\\
&= \hat{Q}\hat{C}\Ket{\psi_q}+\hat{C}\hat{Q}\Ket{\psi_q} \\
&= \hat{Q}\Ket{\psi_{-q}}+q\hat{C}\Ket{\psi_q}\\
&= -q\Ket{\psi_{-q}}+q\Ket{\psi_{-q}}\\
&=0
\end{align*}

If $\hat A\Ket{\psi_q}=0$, then $\hat A\left(\sum a_q\Ket{\psi_q}\right)=0$. We can say that the eigenvalue of $\hat A$ is 0.

For a state $\Ket{\psi_q}$ to be an eigenstate of $\hat C$, $c\Ket{\psi_q}=\hat C\Ket{\psi_q}=\Ket{\psi_{-q}}$

Since $\Ket{\psi_{-q}},\Ket{\psi_q}$ have different eigenvalues (except when $q=0$), they are linearly independent, and thus the  only possible value of $c$ is 0, which isn't an eigenstate.

So the only common eigenstate is $\Ket{\psi_0}$, provided that it is not a null vector.
\section*{Problem 9}
\newcommand{\kpsi}{\Ket{\psi}}
\begin{align*}
[\hat x, \exp\left(\frac{i\hat p a}{\hbar}\right)]\kpsi &= [\hat x, \exp\left(\frac{i(-i)\hbar\partial_x a}{\hbar}\right)]\kpsi\\
&=[\hat x, \exp(a\partial_x)]\kpsi\\
&=  x \sum_i \frac1{i!}a^i\partial^i_x\kpsi - \sum_i\frac1{i!} a^i\partial^i_x (x\kpsi) \\
&=  x \sum_i \frac1{i!}a^i\partial^i_x\kpsi - \sum_i\frac1{i!} a^i(x\partial^i_x (\kpsi) + \partial_x^{i-1}i\kpsi)\\
&= - \sum_i\frac1{i!} a^i \partial_x^{i-1}i\kpsi\\
&= - \sum_i\frac1{(i-1)!} a^i \partial_x^{i-1}\kpsi\\
&= - a\exp(a\partial_x)\Ket{\psi}\\
&= -a\exp\left(ia\frac{\hat p}{\hbar}\right)\kpsi\\
\therefore [\hat x, \exp\left(\frac{i\hat p a}{\hbar}\right)] &= -a\exp\left(ia\frac{\hat p}{\hbar}\right)
\end{align*}

\section*{Problem 11}
The first allowed state (ground state) in the new system will be the first odd wavefunction, i.e. when $n=1$. We need to calculate the probability that the current state ($n=0$) becomes $n=1$ in that region. Note that since the space is halved over a symmetric function, the wavefunctions will be normalized by an extra factor of $\frac1{\sqrt2}$

So, the overlap is 
\begin{align*}
\int_0^\infty \frac1{\sqrt2}\psi_0(x)\frac1{\sqrt2}\psi_1(x)
&= \int_0^\infty \frac1{2} \left(\frac{\alpha}{\sqrt{\pi}}\right)^{\frac12}e^{-\alpha^2x^2/2}H_0(\alpha x)\left(\frac{\alpha}{\sqrt{\pi}2}\right)^\frac12 e^{-\alpha^2x^2/2}H_2(\alpha x)\\
&= \frac1{2}\frac{\alpha}{\sqrt\pi\sqrt2}\cdot \int_0^\infty e^{-\alpha^2x^2}\cdot 2\alpha x\\
&=\frac{\alpha^2}{\sqrt{2\pi}} \int_0^\infty e^{-\alpha^2x^2}\cdot  x\\
&=\frac{\alpha^2}{\sqrt{2\pi}}\frac{1}{2\alpha^2}
&=\frac{1}{2\sqrt\pi}
\end{align*}

Thus the probability is $\frac1{\sqrt{2\pi}}$
\end{document}
