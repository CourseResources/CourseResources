% LaTeX file for a 1 page document
\documentclass[12pt]{article}
\usepackage{amsmath}
\usepackage{amssymb}
\usepackage{amsthm}
\usepackage{physymb}
\usepackage{braket}
\usepackage{array} 
\title{EP 307 Assignment 4}

\author{Manish Goregaokar\\120260006}

\begin{document}
\maketitle

\section*{Problem 1}
$$\prod_{i\neq s}\frac{\hat A-a_i}{a_s-a_i}$$

If $\Ket{a}=\sum c_i\Ket{a_i}$, applying this operator we have \begin{align*}
\prod_{i\neq s}\frac{\hat A-a_i}{a_s-a_i}\left(\sum c_i\Ket{a_i}\right)&=\sum_j\left(\prod_{i\neq s}\frac{\hat A-a_i}{a_s-a_i}\right)c_j\Ket{a_j}\\
&=\left(\prod_{i\neq s}\frac{\hat A-a_i}{a_s-a_i}\right)c_s\Ket{a_s}\\&\qquad\text{Since the other terms have numerator cancelling out for $i=j$}\\
&=\left(\prod_{i\neq s}\frac{a_s-a_i}{a_s-a_i}\right)c_s\Ket{a_s}\\
&=c_s\Ket{a_s}
\end{align*}
Thus the operator is $\frac{\Ket{a_s}\Bra{a_s}}{\Braket{a_s|a_s}}=p_s$ (projection operator)
\section*{Problem 2}
\newcommand{\Tr}{\mathrm{Tr}}
To prove:
$$\Tr(XY)=\Tr(YX)$$

The trace of an operator can be denoted by $\sum_i \Braket{i|A|i}$, where $\Bra{i},\Ket{i}$ are basis vectors corresponding to the $i$th element being $1$ and the rest $0$. 

Thus,

\begin{align*}
\Tr(AB) 
&= \sum_i \Braket{i|AB|i}\\
&= \sum_i  \Braket{i|AIB|i}\\
&= \sum_i\sum_j  \Braket{i|A\Ket{j}\Bra{j}B|i}\\
&= \sum_i\sum_j  \Braket{j|A\Ket{i}\Bra{i}B|j}\\
&= \sum_j  \Braket{j|BIA|j}\\
&= \Tr(BA)
\end{align*}

\section*{Problem 7}

Applying $\hat{A}=\hat{Q}\hat{C}+\hat{C}\hat{Q}$ to an eigenstate $\Ket{\psi_q}$, we get:

\begin{align*}
\hat{A}\Ket{\psi_q} &=(\hat{Q}\hat{C}+\hat{C}\hat{Q})\Ket{\psi_q}\\
&= \hat{Q}\hat{C}\Ket{\psi_q}+\hat{C}\hat{Q}\Ket{\psi_q} \\
&= \hat{Q}\Ket{\psi_{-q}}+q\hat{C}\Ket{\psi_q}\\
&= -q\Ket{\psi_{-q}}+q\Ket{\psi_{-q}}\\
&=0
\end{align*}

If $\hat A\Ket{\psi_q}=0$, then $\hat A\left(\sum a_q\Ket{\psi_q}\right)=0$. We can say that the eigenvalue of $\hat A$ is 0.

For a state $\Ket{\psi_q}$ to be an eigenstate of $\hat C$, $c\Ket{\psi_q}=\hat C\Ket{\psi_q}=\Ket{\psi_{-q}}$

Since $\Ket{\psi_{-q}},\Ket{\psi_q}$ have different eigenvalues (except when $q=0$), they are linearly independent, and thus the  only possible value of $c$ is 0, which isn't an eigenstate.

So the only common eigenstate is $\Ket{\psi_0}$, provided that it is not a null vector.
\section*{Problem 9}
\newcommand{\kpsi}{\Ket{\psi}}
\begin{align*}
[\hat x, \exp\left(\frac{i\hat p a}{\hbar}\right)]\kpsi &= [\hat x, \exp\left(\frac{i(-i)\hbar\partial_x a}{\hbar}\right)]\kpsi\\
&=[\hat x, \exp(a\partial_x)]\kpsi\\
&=  x \sum_i \frac1{i!}a^i\partial^i_x\kpsi - \sum_i\frac1{i!} a^i\partial^i_x (x\kpsi) \\
&=  x \sum_i \frac1{i!}a^i\partial^i_x\kpsi - \sum_i\frac1{i!} a^i(x\partial^i_x (\kpsi) + \partial_x^{i-1}i\kpsi)\\
&= - \sum_i\frac1{i!} a^i \partial_x^{i-1}i\kpsi\\
&= - \sum_i\frac1{(i-1)!} a^i \partial_x^{i-1}\kpsi\\
&= - a\exp(a\partial_x)\Ket{\psi}\\
&= -a\exp\left(ia\frac{\hat p}{\hbar}\right)\kpsi\\
\therefore [\hat x, \exp\left(\frac{i\hat p a}{\hbar}\right)] &= -a\exp\left(ia\frac{\hat p}{\hbar}\right)
\end{align*}

\section*{Problem 11}
The first allowed state (ground state) in the new system will be the first odd wavefunction, i.e. when $n=1$. We need to calculate the probability that the current state ($n=0$) becomes $n=1$ in that region. Note that since the space is halved over a symmetric function, the wavefunctions will be normalized by an extra factor of $\frac1{\sqrt2}$

So, the overlap is 
\begin{align*}
\int_0^\infty \psi_0(x)\frac1{\sqrt2}\psi_1(x)
&= \int_0^\infty  \left(\frac{\alpha}{\sqrt{\pi}}\right)^{\frac12}e^{-\alpha^2x^2/2}H_0(\alpha x)\left(\frac{\alpha}{\sqrt{\pi}2}\right)^\frac12 e^{-\alpha^2x^2/2}H_2(\alpha x)\\
&= \frac{\alpha}{\sqrt\pi\sqrt2}\cdot \int_0^\infty e^{-\alpha^2x^2}\cdot 2\alpha x\\
&=\frac{2\alpha^2}{\sqrt{2\pi}} \int_0^\infty e^{-\alpha^2x^2}\cdot  x\\
&=\frac{2\alpha^2}{\sqrt{2\pi}}\frac{1}{2\alpha^2}
&=\frac{1}{\sqrt{2\pi}}
\end{align*}

Thus the probability is $\left(\frac1{\sqrt{2\pi}}\right)^2=\boxed{\frac{1}{2\pi}}$

\section*{Problem 13}
$$\hat{A}=\begin{pmatrix}a&0&0\\0&-a&0\\0&0&-a\end{pmatrix},\quad\hat{B}=\begin{pmatrix}b&0&0\\0&0&ib\\0&ib&0\end{pmatrix}$$

To find eigenvectors of $\hat B$, we can multiply it with the vector $X=\begin{pmatrix}x\\y\\z\end{pmatrix}$. We get $\hat BX=\begin{pmatrix}bx\\ibz\\iby\end{pmatrix}=cX$, so for an eigenvector we either have $c=b,ibz=y,iby=z$ (giving $y=z=0$ for a general $b$, and eigenvector $\begin{pmatrix}1\\0\\0\end{pmatrix}$. If $c\neq b$ we have $x=0$, and $y=\pm z$ with eigenvalues $\pm ib$

Therefore the eigenvectors are:

\begin{enumerate}
\item $\begin{pmatrix}1\\0\\0\end{pmatrix}$  or $\Ket{1}$ with eigenvalue $b$
\item $\begin{pmatrix}0\\1\\1\end{pmatrix}$ or $\Ket{2}+\Ket{3}$ with eigenvalue $ib$
\item $\begin{pmatrix}0\\1\\-1\end{pmatrix}$ or $\Ket{2}-\Ket{3}$ with eigenvalue $-ib$
\end{enumerate}
\hfill Ans. (a)

Now,
$$\hat A\hat B=\left(
\begin{array}{ccc}
 a & 0 & 0 \\
 0 & -a & 0 \\
 0 & 0 & -a \\
\end{array}
\right) \left(
\begin{array}{ccc}
 b & 0 & 0 \\
 0 & 0 & i b \\
 0 & i b & 0 \\
\end{array}
\right)=\left(
\begin{array}{ccc}
 a b & 0 & 0 \\
 0 & 0 & 0 \\
 0 & 0 & 0 \\
\end{array}
\right)$$

and 

$$\hat B\hat A= \left(
\begin{array}{ccc}
 b & 0 & 0 \\
 0 & 0 & i b \\
 0 & i b & 0 \\
\end{array}
\right)\left(
\begin{array}{ccc}
 a & 0 & 0 \\
 0 & -a & 0 \\
 0 & 0 & -a \\
\end{array}
\right)=\left(
\begin{array}{ccc}
 a b & 0 & 0 \\
 0 & 0 & 0 \\
 0 & 0 & 0 \\
\end{array}
\right)$$

As $\hat A\hat B=\hat B\hat A$ in matrix form, the operators commute.
\hfill Ans. (b)

The eigenkets of $\hat A$ can be easily seen to be $\Ket{1}$ with eigenvalue $a$, and $l\Ket{2}+m\Ket{3}$ with eigenvalue $-a$ ($\forall~l,m$)

We can see that the eigenkets of $B$ are also eigenkets of $A$. They are also orthogonal, after normalizing we have:\\~\\

\begin{tabular}{ >{$}c<{$} >{$}c<{$}  >{$}c<{$} }
\text{Orthonormal Eigenket} & \text{Eigenvalue with }\hat A & \text{Eigenvalue with }\hat B \\
\Ket{1} & b & a\\
\frac{\Ket{2}+\Ket{3}}{\sqrt{2}} & ib & -a\\
\frac{\Ket{2}-\Ket{3}}{\sqrt{2}} & -ib & -a
\end{tabular}


The eigenkets are not completely determined from the eigenvalues from individual eigenvalues, as the eigenvalue $-a$ has multiplicity 2 and thus has an entire space of eigenkets.

However, knowing the eigenvalues from both operators completely specifies the eigenket, barring a constant.
\hfill Ans. (c)
\section*{Problem 16}

$\psi(x)=\frac{1}{\sqrt{2a}}$ in $[-a,a]$. The momentum space function can be found via the fourier transform, \begin{align*}
\tilde{\psi}(p)&=\frac1{\sqrt{2\pi}}\int_{-\infty}^\infty e^{-ikx}\psi(x)dx\\
&= \frac1{\sqrt{2\pi}}\int_{-a}^a\frac{1}{\sqrt{2a}} e^{-ikx}dx\\
&=\frac{i}{2k\sqrt{a\pi}} \left.e^{-ikx}\right|_{-a}^a\\
&=\frac{i}{2k\sqrt{a\pi}} \left(e^{-ika}-e^{ika}\right)\\
&=\frac{i}{2k\sqrt{a\pi}} \cdot -2i\sin ka\\
\therefore \tilde{\psi}(p)&=\frac{\sin \left(\frac{ap}{\hbar} \right)}{p\sqrt{a\pi}}\qquad\text{(Already normalized)}
\end{align*}
\hfill Ans.

Now, \begin{align*}\sigma_x&=\sqrt{\Braket{x^2}-\Braket{x}^2}\\
&=\sqrt{\int \psi^*(x)x^2\psi(x)dx-0}\\
&= \sqrt{\int_{-a}^a \left(\frac{1}{\sqrt{2 a}}\right)^2 x^2 dx}\\
&= \frac{a}{\sqrt{3}}
\end{align*}

And

\begin{align*}\sigma_p&=\sqrt{\Braket{p^2}-\Braket{p}^2}\\
&=\sqrt{\int \tilde{\psi}^*(p)p^2\tilde{\psi}(p)dp-0}\\
&= \sqrt{\int_{-\infty}^\infty \left(\frac{\sin \left(\frac{a p}{\hbar }\right)}{\sqrt{\pi } \sqrt{a} p}\right)^2 p^2 dp - 0}\qquad\text{(The wavefunction is odd so the second term vanishes)}\\
&=\sqrt{\infty}
\end{align*}

The product of the two uncertantainties is greater than $\frac\hbar2$
\section*{Problem 17}

$$\psi(x,t)=\frac{1}{\sqrt{2 \pi  \hbar }}\int_{-\infty }^{\infty }  \phi(p)  e^{-\frac{t \left(i p^2\right)}{2 m}+i p x} \, dp$$

Now, we can rewrite this as $\frac{1}{\sqrt{2 \pi  \hbar }}\int_{-\infty }^{\infty } e^{i p x} \left(\phi(p)  e^{-\frac{t \left(i p^2\right)}{2 m}}\right) \, dp$

which is an inverse fourier transform.

Thus, $\tilde{\psi}(p,t)=\mathcal F(\psi(x,t))=\phi(p)  e^{-\frac{t \left(i p^2\right)}{2 m}}$\\
Now,
\begin{align*}
\Braket{p}&=\int \tilde{\psi}^*(p,t)p\tilde{\psi}(p,t)dp\\
&=\int \phi^*(p)  e^{\frac{t \left(i p^2\right)}{2 m}}p\phi(p)  e^{-\frac{t \left(i p^2\right)}{2 m}}dp\\
&= \int \phi^*(p) p \phi(p) dp
\end{align*}

Thus, $\Braket{\hat p}$ is independant of time\hfill Ans. (a)

Note that $\phi(p)=\tilde{\psi}(p,t=0)$
Now, 

\begin{align*}
\Braket{x}= \int \tilde{\psi}^*(p,t)i\hbar\ud{}{p}\tilde{\psi}(p,t)dp\\
&=\int \phi^*(p)e^{\frac{t \left(i p^2\right)}{2 m}}i\hbar\left(\pd{\phi}{p}e^{-\frac{t \left(i p^2\right)}{2 m}}+\phi(p)\frac{2pt}{2m}e^{-\frac{t \left(i p^2\right)}{2 m}}\right)dp\\
&=\int\left( \phi^*(p)i\hbar\pd{\phi}{p} + \frac{t}{m}\phi^*(p)p\phi(p)\right) dp\\
&= \Braket{i\hbar\pd{}{p}}_{t=t_0} + \frac{t}{m}\Braket{p}_{t=t_0}\\
\therefore \Braket{x}&=\Braket{x}_{t=0} + \frac{t}{m}\Braket{p}_{t=0}
\end{align*}

Since this is linear, we can shift by $t_0$ to get $\boxed{\Braket{x}=\Braket{x}_{t=t_0} + \frac{t-t_0}{m}\Braket{p}_{t=t_0}}$\hfill Ans. (b)
\section{Problem 19}
$$\hat H = \frac{\hat p^2}{2m}+V(x)$$

\begin{align*}
[H,\hat x]&=\frac{\hat p^2}{2m}x+V(x)x-x\frac{\hat p^2}{2m}+xV(x)\\
&=-x\frac{\hat p^2}{2m}\\
[[H,\hat x],x]&=-x\frac{\hat p^2}{2m}x - -x^2\frac{\hat p^2}{2m}\\
&=\frac{x^2\hat p^2}{2m}
\end{align*}
\end{document}
