% LaTeX file for a 1 page document
\documentclass[12pt]{article}
\usepackage{amsmath}
\usepackage{amssymb}
\usepackage{amsthm}
\usepackage{physymb}
\usepackage{braket}
\usepackage{array} 
\title{EP 307 Assignment 6}

\author{Manish Goregaokar\\120260006}
\date{March 19, 2014}
\begin{document}
\maketitle
\newcommand{\ad}{{\hat a^\dagger}}
\newcommand{\ah}{\hat a}
\section*{Problem 2}	

To prove: $\Braket{n|T|n}=\Braket{n|V|n}$

Firstly, we note that $\hat x =\sqrt{\frac{\hbar}{2m\omega}}(\ah+\ad), \hat p =i\sqrt{\frac{m \omega\hbar}{2}}(\ad-\ah)$


\begin{align*}
\therefore \Braket{n|T|n} &= \Braket{n|\frac1{2m}\hat p^2|n}\\
&= -\frac{\omega\hbar}{4} \Braket{n|(\ad -\ah)^2|n}\\
\text{And}~~ \Braket{n|V|n} &= \Braket{n|\frac12m\omega^2\hat x^2|n}\\
&= \frac{\omega\hbar}{4} \Braket{n|(\ad +\ah)^2|n}\\
\therefore \Braket{n|V|n} - \Braket{n|T|n} &=  \frac{\omega\hbar}{4} \Braket{n|(\ad +\ah)^2|n} + \frac{\omega\hbar}{4} \Braket{n|(\ad -\ah)^2|n}\\
&= \frac{\omega\hbar}{2} \Braket{n|{\ad}^2 +{\ah}^2|n}\\
&=0
\end{align*}

since both $\ad$ will move the state to $\Ket{n+2}$, which is orthogonal to $\Ket{n}$, and $\ah$ will move it to $\Ket{n-2}$ or $0\Ket{0}$ (if $n<2$). In both cases the net result is zero

Thus  $\Braket{n|T|n}=\Braket{n|V|n}$

\section*{Problem 3}

We know that $\hat x =\sqrt{\frac{\hbar}{2m\omega}}(\ah+\ad), \hat p =i\sqrt{\frac{m \omega\hbar}{2}}(\ad-\ah)$, and furthermore $\ah\Ket{\alpha}=\alpha\Ket{\alpha}$. Taking the conjugate, $\Bra{\alpha}\ad=\alpha*\Bra{\alpha}$

Now,\begin{align*}
\Braket{x} &= \sqrt{\frac{\hbar}{2m\omega}} \Braket{\alpha|\ah+\ad|\alpha}\\
&= \sqrt{\frac{\hbar}{2m\omega}}(\Braket{\alpha|\ah|\alpha}+ \Braket{\alpha|\ad|\alpha})\\
&=\sqrt{\frac{\hbar}{2m\omega}}(\Braket{\alpha|\alpha|\alpha}+ \alpha^*\Braket{\alpha|\alpha})\\
&= \sqrt{\frac{\hbar}{2m\omega}}(\alpha+\alpha^*)\\
\Braket{x^2} &= \frac{\hbar}{2m\omega} \Braket{\alpha|(\ah+\ad)^2|\alpha}\\
&=\frac{\hbar}{2m\omega}(\Braket{\alpha|\ah^2|\alpha}+ \Braket{\alpha|\ad^2|\alpha}+\Braket{\alpha|\ah\ad|\alpha}+\Braket{\alpha|\ad\ah|\alpha})\\
&=\frac{\hbar}{2m\omega}(\Braket{\alpha|\alpha^2|\alpha}+ (\alpha^*)^2\Braket{\alpha|\alpha}+\Braket{\alpha|\ad\ah+[\ah,\ad]|\alpha}+\Braket{\alpha|\ad\ah|\alpha})\\
&=\frac{\hbar}{2m\omega}(\alpha^2 *{\alpha^*}^2+\Braket{\alpha|\ad\ah+1|\alpha}+\Braket{\alpha|\ad\ah|\alpha})\\
&=\frac{\hbar}{2m\omega}(\alpha^2+ {\alpha^*}^2+\Braket{\alpha|1|\alpha}+2\Braket{\alpha|\ad\ah|\alpha})\\
&=\frac{\hbar}{2m\omega}(\alpha^2+ {\alpha^*}^2+1+2\alpha^*\alpha)\\
\Braket{p} &=i\sqrt{\frac{m \omega\hbar}{2}}\Braket{\alpha|(\ad-\ah)|\alpha}\\
&=i\sqrt{\frac{m \omega\hbar}{2}}(\alpha^*-\alpha)\\
\Braket{p^2} &=-\frac{m \omega\hbar}{2}\Braket{\alpha|(\ad-\ah)^2|\alpha}\\
&=-\frac{m \omega\hbar}{2}({\alpha^*}^2+\alpha^2-1+2\alpha^*\alpha)\\
\therefore \sigma_x &=\sqrt{\Braket{x^2}-\Braket{x}^2}\\
&= \sqrt{\frac{\hbar}{2m\omega}\left(\alpha^2+ {\alpha^*}^2+1+2\alpha^*\alpha-(\alpha+\alpha^*)^2\right)}\\
&= \sqrt{\frac{\hbar}{2m\omega}}\\
\therefore \sigma_p &=\sqrt{\Braket{p^2}-\Braket{p}^2}\\
&= \sqrt{-\frac{m \omega\hbar}{2}\left(\alpha^2+ {\alpha^*}^2-1-2\alpha\alpha^*-(\alpha-\alpha^*)^2\right)}\\
&= \sqrt{-\frac{m \omega\hbar}{2}\left(-1\right)}\\
&=\sqrt{\frac{m \omega\hbar}{2}}\\
\therefore \sigma_x\sigma_p &= \sqrt{\frac{\hbar}{2m\omega}}\sqrt{\frac{m \omega\hbar}{2})}\\
&= \frac{\hbar}{2}
\end{align*}

Thus this system fllows Heisenberg's uncertainty relation.
\section*{Problem 4}
\section*{Problem 5}
\section*{Problem 6}

\end{document}
