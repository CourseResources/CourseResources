% LaTeX file for a 1 page document
\documentclass[12pt]{article}
\usepackage{amsmath}
\usepackage{amssymb}
\usepackage{amsthm}
\usepackage{physymb}
\usepackage{braket}
\usepackage{array} 
\title{EP 307 Assignment 6}

\author{Manish Goregaokar\\120260006}
\date{March 19, 2014}
\begin{document}
\maketitle
\newcommand{\ad}{{\hat a^\dagger}}
\newcommand{\ah}{\hat a}
\section*{Problem 2}	

To prove: $\Braket{n|T|n}=\Braket{n|V|n}$

Firstly, we note that $\hat x =\sqrt{\frac{\hbar}{2m\omega}}(\ah+\ad), \hat p =i\sqrt{\frac{m \omega\hbar}{2}}(\ad-\ah)$


\begin{align*}
\therefore \Braket{n|T|n} &= \Braket{n|\frac1{2m}\hat p^2|n}\\
&= -\frac{\omega\hbar}{4} \Braket{n|(\ad -\ah)^2|n}\\
\text{And}~~ \Braket{n|V|n} &= \Braket{n|\frac12m\omega^2\hat x^2|n}\\
&= \frac{\omega\hbar}{4} \Braket{n|(\ad +\ah)^2|n}\\
\therefore \Braket{n|V|n} - \Braket{n|T|n} &=  \frac{\omega\hbar}{4} \Braket{n|(\ad +\ah)^2|n} + \frac{\omega\hbar}{4} \Braket{n|(\ad -\ah)^2|n}\\
&= \frac{\omega\hbar}{2} \Braket{n|{\ad}^2 +{\ah}^2|n}\\
&=0
\end{align*}

since both $\ad$ will move the state to $\Ket{n+2}$, which is orthogonal to $\Ket{n}$, and $\ah$ will move it to $\Ket{n-2}$ or $0\Ket{0}$ (if $n<2$). In both cases the net result is zero

Thus  $\Braket{n|T|n}=\Braket{n|V|n}$
\section*{Problem 3}
Such a linear combination would be $\frac1{\sqrt{1+\eta^2}}(\Ket{1}+\eta\Ket{0})$
\begin{align*}
\Braket{x} &= \frac1{1+\eta^2}(\Bra{1}+\eta\Bra{0})\left(\sqrt{\frac{\hbar}{2m\omega}}(\ah+\ad)\right)(\Ket{1}+\eta\Ket{0})\\
&=\sqrt{\frac{\hbar}{2m\omega}}\frac1{1+\eta^2}(\Bra{1}+\eta\Bra{0})\left(\ah\Ket{1}+\eta\ah\Ket{0}+\ad\Ket{1}+\eta\ad\Ket{0}\right)\\
&=\sqrt{\frac{\hbar}{2m\omega}}\frac1{1+\eta^2}(\Bra{1}+\eta\Bra{0})\left(\Ket{0}+0+\sqrt2\Ket{2}+\eta\Ket{1}\right)\\
&=\sqrt{\frac{\hbar}{2m\omega}}\frac1{1+\eta^2}(2\eta)
\end{align*}
Maximizing $\frac{\eta}{1+\eta^2}$ we get $\eta=1$, and thus the state is $\frac{\Ket{1}+\Ket{0}}{\sqrt2}$
\section*{Problem 4}

We know that $\hat x =\sqrt{\frac{\hbar}{2m\omega}}(\ah+\ad), \hat p =i\sqrt{\frac{m \omega\hbar}{2}}(\ad-\ah)$, and furthermore $\ah\Ket{\alpha}=\alpha\Ket{\alpha}$. Taking the conjugate, $\Bra{\alpha}\ad=\alpha^*\Bra{\alpha}$

Now,\begin{align*}
\Braket{x} &= \sqrt{\frac{\hbar}{2m\omega}} \Braket{\alpha|\ah+\ad|\alpha}\\
&= \sqrt{\frac{\hbar}{2m\omega}}(\Braket{\alpha|\ah|\alpha}+ \Braket{\alpha|\ad|\alpha})\\
&=\sqrt{\frac{\hbar}{2m\omega}}(\Braket{\alpha|\alpha|\alpha}+ \alpha^*\Braket{\alpha|\alpha})\\
&= \sqrt{\frac{\hbar}{2m\omega}}(\alpha+\alpha^*)\\
\Braket{x^2} &= \frac{\hbar}{2m\omega} \Braket{\alpha|(\ah+\ad)^2|\alpha}\\
&=\frac{\hbar}{2m\omega}(\Braket{\alpha|\ah^2|\alpha}+ \Braket{\alpha|\ad^2|\alpha}+\Braket{\alpha|\ah\ad|\alpha}+\Braket{\alpha|\ad\ah|\alpha})\\
&=\frac{\hbar}{2m\omega}(\Braket{\alpha|\alpha^2|\alpha}+ (\alpha^*)^2\Braket{\alpha|\alpha}+\Braket{\alpha|\ad\ah+[\ah,\ad]|\alpha}+\Braket{\alpha|\ad\ah|\alpha})\\
&=\frac{\hbar}{2m\omega}(\alpha^2 *{\alpha^*}^2+\Braket{\alpha|\ad\ah+1|\alpha}+\Braket{\alpha|\ad\ah|\alpha})\\
&=\frac{\hbar}{2m\omega}(\alpha^2+ {\alpha^*}^2+\Braket{\alpha|1|\alpha}+2\Braket{\alpha|\ad\ah|\alpha})\\
&=\frac{\hbar}{2m\omega}(\alpha^2+ {\alpha^*}^2+1+2\alpha^*\alpha)\\
\Braket{p} &=i\sqrt{\frac{m \omega\hbar}{2}}\Braket{\alpha|(\ad-\ah)|\alpha}\\
&=i\sqrt{\frac{m \omega\hbar}{2}}(\alpha^*-\alpha)\\
\Braket{p^2} &=-\frac{m \omega\hbar}{2}\Braket{\alpha|(\ad-\ah)^2|\alpha}\\
&=-\frac{m \omega\hbar}{2}({\alpha^*}^2+\alpha^2-1+2\alpha^*\alpha)\\
\therefore \sigma_x &=\sqrt{\Braket{x^2}-\Braket{x}^2}\\
&= \sqrt{\frac{\hbar}{2m\omega}\left(\alpha^2+ {\alpha^*}^2+1+2\alpha^*\alpha-(\alpha+\alpha^*)^2\right)}\\
&= \sqrt{\frac{\hbar}{2m\omega}}\\
\therefore \sigma_p &=\sqrt{\Braket{p^2}-\Braket{p}^2}\\
&= \sqrt{-\frac{m \omega\hbar}{2}\left(\alpha^2+ {\alpha^*}^2-1-2\alpha\alpha^*-(\alpha-\alpha^*)^2\right)}\\
&= \sqrt{-\frac{m \omega\hbar}{2}\left(-1\right)}\\
&=\sqrt{\frac{m \omega\hbar}{2}}\\
\therefore \sigma_x\sigma_p &= \sqrt{\frac{\hbar}{2m\omega}}\sqrt{\frac{m \omega\hbar}{2})}\\
&= \frac{\hbar}{2}
\end{align*}

Thus this system follows Heisenberg's uncertainty relation.\hfill Ans (a)

\begin{align*}
\Braket{N} &= \Braket{\alpha|\ad\ah|\alpha}\\
&=\alpha\Braket{\alpha|\alpha|\alpha}\\
&=\alpha^2
\end{align*}

The probability amplitude is $\Braket{n|\alpha}$

\begin{align*}
\Braket{n|\alpha} &= \Braket{n|e^{-\frac{\left|\alpha\right|^2}{2}}e^{\alpha\ad}|0}\\
&=e^{-|\alpha|^2/2}\Braket{n|1 + \alpha\ad+\frac12(\alpha\ad)^2 + ...|0}\\
&=e^{-|\alpha|^2/2}\Bra{n}\left(1 + \alpha\Ket{1}+\frac12(\alpha)^2\sqrt2\Ket{2} +...\right)\\
&= e^{-|\alpha|^2/2}\frac{\alpha^n}{n!}\sqrt{n!}
\end{align*}

Thus the probability is $e^{-|\alpha|^2}\frac{\alpha^2n}{n!}$. This is a Poisson distribution, due to the $\frac{\alpha^2n}{n!}$ factor.\hfill Ans (b)

Since the displacement operator is equivalent to $e^{\alpha\ad-\alpha^*\ah}$, its conjugate is $e^{-\alpha\ad+\alpha^*\ah}$ (conjugating all of the components).

Also, $e^{\hat A}e^{-\hat A}=e^{\hat A-\hat A}e^{[\hat A,-\hat A]}1$.

Thus, $DD^\dagger = 1$, for $\hat A=\alpha\ad-\alpha^*\ah$. Thus it is unitary
\section*{Problem 5}
\section*{Problem 6} 
\begin{align*}
(\vec \sigma\cdot \vec a)(\vec \sigma \cdot \vec b) &= \sigma_ia_i\sigma_jb_j\\
&=a_ib_j\sigma_i\sigma_j\\
&=a_ib_j(\delta_{ij}+\epsilon_{ijk}\sigma_k)\\
&=a_ib_i + \epsilon_{ijk}\sigma_ka_ib_j\\
&=\vec a \cdot\vec b + \vec \sigma\cdot(\vec a \times \vec b)\qquad\qquad\text{Ans. (a)}
\end{align*}

\begin{align*}
e^{i\vec a \cdot \sigma}&=e^{ia\hat n \cdot \sigma}\\
&=1+ ia\hat n \cdot\hat \sigma + \frac12i^2a^2 (\hat n \cdot\hat \sigma)^2 + ...\\
\end{align*}
For even terms, the  $(\hat n \cdot\hat \sigma)^{2n}$ is the identity, since \begin{align*}(\hat n \cdot\hat \sigma)^{2}&=(\hat n \cdot\hat \sigma)(\hat n \cdot\hat \sigma)\\ &= \hat n\cdot\hat n + \sigma\cdot (\hat n \times\hat n)\\&=I\end{align*}. For odd terms, a single $\hat n \cdot\hat \sigma$ remains.

Collecting the even terms we get $\cos (a) I$, and for the odd terms we get $i\sin(a)\hat n \cdot\hat \sigma$.

Thus, the final expression is $I\cos a + i\frac{\vec a \cdot \sigma}{a}\sin a$\hfill Ans. (b)
\end{document}
