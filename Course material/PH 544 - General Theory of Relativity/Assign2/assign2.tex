\documentclass[12pt]{article}
\usepackage{amsmath}
\usepackage{amsthm}
\usepackage{physymb}
\usepackage{enumerate}

\author{Manish Goregaokar\\120260006}
\title{Tutorial 2}
\begin{document}
\maketitle

\section*{Problem 1}
\begin{enumerate}[(i)]
\item $\ud{x^\mu}{\tau}$ is a tensor for all kinds of transformations. $\ud{x'^\mu}{\tau}=\ud{}{\tau}\left(x^\nu\ud{x'^\mu}{x^\nu}\right)=\ud{x'^\mu}{x^\nu}\ud{x^\nu}{\tau}$, and thus we get a contravariant tensor
\item $\udd{x^\mu}{\tau}$ is also a tensor because it is the derivative of a tensor (Use the property proved in $(i)$ by replacing $x^\mu$ with $\ud{x^\mu}{\tau}$). This tensor is also contravariant.
\item $\partial_\mu\phi$ is a tensor. $\partial_\mu'\phi=\pd{\phi}{ x_\nu\ud{x_\mu'}{x_\nu}}=\ud{x_\nu}{x_\mu'}\pd{\phi}{x_\nu}=\ud{x_\nu}{x_\mu'}\partial_\mu\phi$, making it a covariant tensor.
\item This is only a tensor when special relativistic constraints are present. We know that the covariant derivative of the tensor $F$, $F_{\mu\nu;\rho}=\partial_\rho F_{\mu\nu}-\Gamma_{\nu\rho}^{\sigma}A_{\mu\sigma}
-\Gamma_{\mu\rho}^{\sigma}A_{\nu\sigma}$ is a tensor itself. However, $\Gamma_{\nu\rho}^{\sigma}F_{\mu\sigma},\Gamma_{\mu\rho}^{\sigma}F_{\nu\sigma}$ are not  tensors in the general case as proved in class, so $\partial_\rho F_{\mu\nu}$ is not. However, in special relativity the Christoffel symbols are zero, so the remaining portion of the equation must be a tensor too.
\end{enumerate}
\section*{Problem 2}
We first calculate the velocity and then $\gamma_v$ of the particle in the frame:

\begin{align*}v^2&=\dot{x}^2+\dot{y}^2+\dot{x^2}\\
&=(a+b\omega\cos\omega t)^2 + (-b\omega\sin\omega t)^2\\
&= a^2 +b^2\omega^2 + 2ab\omega\cos\omega t\\
\therefore \gamma_v &= \frac{1}{\sqrt{1-\frac{a^2 +b^2\omega^2 + 2ab\omega\cos\omega t}{c^2}}}
\end{align*}

From this, we get $w^\mu=\frac{1}{\sqrt{1-\frac{a^2 +b^2\omega^2 + 2ab\omega\cos\omega t}{c^2}}}\begin{bmatrix}

a+b\omega\cos\omega t\\
-b\omega\sin\omega t\\
0
\end{bmatrix}$

\section*{Problem 3}
$$t=\frac{X}{a}\sinh aT, x\frac{X}{a}\cosh a T$$
\begin{align*}
ds^2&=dx^2-dt^2\\
&= (\frac1{a}\sinh aTdX + \frac{X}{a}\cosh aT dT)^2
\end{align*}

\end{document}