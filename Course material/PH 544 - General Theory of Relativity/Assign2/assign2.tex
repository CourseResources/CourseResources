\documentclass[12pt]{article}
\usepackage{amsmath}
\usepackage{amsthm}
\usepackage{physymb}
\usepackage{enumerate}

\author{Manish Goregaokar\\120260006}
\title{Tutorial 2}
\begin{document}
\maketitle

\section*{Problem 1}
\begin{enumerate}[(i)]
\item $\ud{x^\mu}{\tau}$ is a tensor for all kinds of transformations. $\ud{x'^\mu}{\tau}=\ud{}{\tau}\left(x^\nu\ud{x'^\mu}{x^\nu}\right)=\ud{x'^\mu}{x^\nu}\ud{x^\nu}{\tau}$, and thus we get a contravariant tensor
\item $\udd{x^\mu}{\tau}$ is also a tensor because it is the derivative of a tensor (Use the property proved in $(i)$ by replacing $x^\mu$ with $\ud{x^\mu}{\tau}$). This tensor is also contravariant.
\item $\partial_\mu\phi$ is a tensor. $\partial_\mu'\phi=\pd{\phi}{ x_\nu\ud{x_\mu'}{x_\nu}}=\ud{x_\nu}{x_\mu'}\pd{\phi}{x_\nu}=\ud{x_\nu}{x_\mu'}\partial_\mu\phi$, making it a covariant tensor.
\item This is only a tensor when special relativistic constraints are present. We know that the covariant derivative of the tensor $F$, $F_{\mu\nu;\rho}=\partial_\rho F_{\mu\nu}-\Gamma_{\nu\rho}^{\sigma}A_{\mu\sigma}
-\Gamma_{\mu\rho}^{\sigma}A_{\nu\sigma}$ is a tensor itself. However, $\Gamma_{\nu\rho}^{\sigma}F_{\mu\sigma},\Gamma_{\mu\rho}^{\sigma}F_{\nu\sigma}$ are not  tensors in the general case as proved in class, so $\partial_\rho F_{\mu\nu}$ is not. However, in special relativity the Christoffel symbols are zero, so the remaining portion of the equation must be a tensor too.
\end{enumerate}
\section*{Problem 2}
We first calculate the velocity and then $\gamma_v$ of the particle in the frame:

\begin{align*}v^2&=\dot{x}^2+\dot{y}^2+\dot{x^2}\\
&=(a+b\omega\cos\omega t)^2 + (-b\omega\sin\omega t)^2\\
&= a^2 +b^2\omega^2 + 2ab\omega\cos\omega t\\
\therefore \gamma_v &= \frac{1}{\sqrt{1-\frac{a^2 +b^2\omega^2 + 2ab\omega\cos\omega t}{c^2}}}
\end{align*}

From this, we get $w^\mu=\frac{1}{\sqrt{1-\frac{a^2 +b^2\omega^2 + 2ab\omega\cos\omega t}{c^2}}}\begin{bmatrix}

a+b\omega\cos\omega t\\
-b\omega\sin\omega t\\
0\\
c
\end{bmatrix}$

The acceleration four vector can be calculated by differentiating this. 

\section*{Problem 3}
$$t=\frac{X}{a}\sinh aT, x\frac{X}{a}\cosh a T$$
\begin{align*}
ds^2&=dx^2-dt^2\\
&= (\sinh aTdX + X\cosh aT dT)^2 - (X\sinh aTdT + \cosh aT dX)^2 \\
&= -dX^2 + dT^2 X^2
\end{align*}
This gives us the metric

$$g_{\mu\nu}\equiv\begin{pmatrix}
X^2 & 0\\ 0 & -1
\end{pmatrix}$$

The inverse is 
$$g^{\mu\nu}\equiv\begin{pmatrix}
\frac1{X^2} & 0\\ 0 & -1
\end{pmatrix}$$

Now, $\Gamma^\mu_{\rho\sigma}=g^{\mu\nu}\frac12(\partial_\sigma g_{\rho\nu}+\partial_\rho g_{\sigma\nu}-\partial_\nu g_{\rho\sigma})$

\begin{align*}
\Gamma^\mu_{11}&=g^{\mu\nu}\frac12(\partial_1 g_{1\nu}+\partial_1 g_{1\nu}-\partial_\nu g_{11})\\
&=g^{\mu\nu}\frac12(0+0-\partial_\nu g_{11})\qquad&\because \text{ g is independant of $T$}\\
&=0\\
\Gamma^\mu_{10}=\Gamma^\mu_{01}&=g^{\mu\nu}\frac12(\partial_0 g_{1\nu}+\partial_1 g_{0\nu}-\partial_\nu g_{10})\\
&=g^{\mu\nu}\frac12(\partial_1 g_{0\nu}-\partial_\nu g_{10})\\
&=g^{\mu0}\frac12(\partial_1 g_{00}-\partial_0 g_{10})+g^{\mu1}\frac12(\partial_1 g_{01}-\partial_1 g_{10})\\
&=-g^{\mu0}\frac1{2X}\\
\Gamma^\mu_{00}&=g^{\mu\nu}\frac12(\partial_0 g_{0\nu}+\partial_0 g_{0\nu}-\partial_\nu g_{00})\\
&=g^{\mu\nu}\frac12(-\partial_\nu g_{00})\\
&=g^{\mu1}\frac12(-\partial_1 g_{00})\\
&=g^{\mu1}\frac1{2X}\\
\end{align*}
$$\therefore\Gamma^\mu_{\rho\sigma}=\begin{pmatrix}
0 & -g^{\mu0}\frac1{2X}\\
-g^{\mu0}\frac1{2X} & g^{\mu1}\frac1{2X}
\end{pmatrix}$$

The equation of motion is $\udd{x^\mu}{\tau}+\Gamma^\mu_{\rho\sigma}\ud{x^\rho}{\tau}\ud{x^\sigma}{\tau}=0$

For $\mu=0$, we have \begin{align*}0&=\udd{T}{\tau}+\Gamma^0_{\rho\sigma}\ud{x^\rho}{\tau}\ud{x^\sigma}{\tau}\\
&=\udd{T}{\tau}-2g^{00}\frac1{2X}\ud{X}{\tau}\ud{T}{\tau}\\
&=\udd{T}{\tau}+X\ud{X}{\tau}\ud{T}{\tau}
\end{align*}
For $\mu=1$, we have \begin{align*}0&=\udd{X}{\tau}+\Gamma^1_{\rho\sigma}\ud{x^\rho}{\tau}\ud{x^\sigma}{\tau}
&= \udd{X}{\tau} + g^{11}\frac1{2X}\left(\ud{X}{\tau}\right)^2\\
&= \udd{X}{\tau} - \frac1{2X}\left(\ud{X}{\tau}\right)^2
\end{align*}

Thus the geodesic trajectory is \begin{align*}
\udd{X}{\tau} &=\frac1{2X}\left(\ud{X}{\tau}\right)^2\\
\udd{T}{\tau}&=-X\ud{X}{\tau}\ud{T}{\tau}
\end{align*}

\section*{Problem 4}
$$x=\mu\nu\qquad y=\frac12(\mu^2-\nu^2)$$
\newcommand{\dd}{\mathrm{d}}
We have 

\begin{align*}
\dd s^2&=\dd x^2+\dd y^2\\
&=(\mu\dd\nu +\nu \dd\mu)^2 + (\mu\dd\mu-\nu\dd\nu)^2\\
&=\dd\mu ^2 \nu ^2+2 \dd\mu \dd\nu \mu  \nu +\dd \nu^2 \mu ^2 + \dd\mu^2 \mu ^2-2\dd\mu\dd\nu \mu  \nu +\dd\nu^2 \nu ^2\\
&=(\mu^2+\nu^2)(\dd\mu^2+\dd\nu^2)
\end{align*}

giving us the new metric $$g_{\mu\nu}\equiv\begin{pmatrix}
\mu^2+\nu^2 & 0\\
0 & \mu^2+\nu^2
\end{pmatrix}$$

\section*{Problem 5}

(Summation notation not used)

$$x^{\mu'}=\frac1{\epsilon_\mu + \frac1{x^\mu}}$$

\begin{align*}x^{\mu'}-x^\mu&=\frac1{\epsilon_\mu + \frac1{x^\mu}} - x^\mu\\
&=\frac{1-\epsilon_\mu x^\mu -1}{\epsilon_\mu + \frac1{x^\mu}}
\end{align*}

Taking limits, we get $x^{\mu'}-x^\mu=-\epsilon_\mu(x^\mu)^2$ (not expected answer)

\section*{Problem 6}

(Some simplification done with the help of {\em Mathematica})


We have the metric as $$g_{\mu\nu}\equiv\begin{pmatrix}
-1 &0&0&0\\
0&1&0&0\\
0&0&b^2+r^2&0\\
0&0&0&(b^2+r^2)\sin^2\theta
\end{pmatrix}$$

The inverse is 

$$g^{\mu\nu}\equiv\begin{pmatrix}
-1 &0&0&0\\
0&1&0&0\\
0&0&\frac1{b^2+r^2}&0\\
0&0&0&\frac1{(b^2+r^2)\sin^2\theta}
\end{pmatrix}$$

We can calculate the Christoffel tensor via $\Gamma^\mu_{\rho\sigma}=g^{\mu\nu}\frac12(\partial_\sigma g_{\rho\nu}+\partial_\rho g_{\sigma\nu}-\partial_\nu g_{\rho\sigma})$

We get $$\Gamma = \left( \begin{array}{cccc}  (0,0,0,0) & (0,0,0,0) & (0,0,0,0) & (0,0,0,0) \\  (0,0,0,0) & (0,0,0,0) & \left(0,0,\frac{2 r}{b^2+r^2},0\right) & \left(0,0,0,\frac{2 r}{b^2+r^2}\right) \\  (0,0,0,0) & \left(0,0,\frac{2 r}{b^2+r^2},0\right) & (0,-2 r,0,0) & (0,0,0,2 \cot (\theta )) \\  (0,0,0,0) & \left(0,0,0,\frac{2 r}{b^2+r^2}\right) & (0,0,0,2 \cot (\theta )) & \left(0,-2 r \sin ^2(\theta ),-2 \cos (\theta ) \sin (\theta ),0\right) \\ \end{array} \right)$$

where the row and column of the outer matrix correspond to $\rho$,$\sigma$, and $\mu$ differentiates the components of the inner matrix.

The geodesic equation is $\udd{x^\mu}{\tau}+\Gamma^\mu_{\rho\sigma}\ud{x^\rho}{\tau}\ud{x^\sigma}{\tau}=0$

Solving it, we get

\begin{align*}
0&=\frac{\dd ^2t}{\dd \tau ^2}\\ 0 &= -2 r \sin ^2(\theta ) \left(\frac{\dd \phi }{\dd \tau }\right)^2-2 r \left(\frac{\dd \theta }{\dd \tau }\right)^2+\frac{\dd ^2r}{\dd \tau ^2}\\ 0&= \frac{4 r \frac{\dd \theta }{\dd \tau } \frac{\dd r}{\dd \tau }}{b^2+r^2}+\frac{\dd ^2\theta }{\dd \tau ^2}-2 \sin (\theta ) \cos (\theta ) \left(\frac{\dd \phi }{\dd \tau }\right)^2\\ 0&=\frac{\dd \phi }{\dd \tau } \left(\frac{2 r \frac{\dd r}{\dd \tau }}{b^2+r^2}+2 \cot (\theta ) \frac{\dd \theta }{\dd \tau }\right)+\frac{2 r \frac{\dd r}{\dd \tau } \frac{\dd \phi }{\dd \tau }}{b^2+r^2}+2 \cot (\theta ) \frac{\dd \theta }{\dd \tau } \frac{\dd \phi }{\dd \tau }+\frac{\dd ^2\phi }{\dd \tau ^2}
\end{align*}
\end{document}