% LaTeX file for a 1 page document
\documentclass[12pt]{article}
\usepackage{amsmath}
\usepackage{amssymb}
\usepackage{amsthm}
%\usepackage{physymb}
\usepackage{array}
\title{MA 207 scribbles}

\author{Manish}

\begin{document}
\maketitle

\section{Series}
\begin{itemize}
\item If A series covnerges then terms tend to zero
\item $\sum f(n)$ converges iff $\int\limits_1^\infty$ converges
\item abs convergent $\implies$ convergent.
\item Alternating ($\pm$) the terms of a monotone decr series is convergent
\item (Cond convgt = convegent but not absolutely). Such series can be rearranged to any preassigned value. Absolutely convergent series have no change on rearangement.
\item Cauchy product converges if both converge and atleast one is absolutely convergent
\end{itemize}
\section{Polynomial-type ODEs}

\begin{itemize}
\item Legendre: $(1 -x^2)y'' - 2xy' + p(p + 1)y = 0$
\item Tchebychev: $(1-x^2)y'' - xy'+p^2y=0$ (orthogonal on $(-1,1)$ with wieght $(1-x^2)^{-0.5}$)
\item Bessel: $x^2y'' +xy' + (x^2-p^2)y=0$
\item Laguerre: $xy'' + (1-x)y' + py=0$
\item Hypergeometric: $x(1-x)y'' + (c- (a+b+1)x)y'-aby=0$
\item (others)
\item Hermite: $y''-2xy'+2ny=0$ (Orthogonal on $\mathbb R$ wrt $e^{-x^2}$)
\item Bernoulli: $B_n'=nB_{n-1}$, $\int\limits_0^1 B_n=0, n\geq 1$
\end{itemize}

If ${f_n}$ is a seq of orth pol, then zeroes are real+distinct (and are in the interval of orthogonality); it satisfies a 3term recursion formula, and the zeroes interlace.

\subsection{Legendre}

Gen soln: \begin{align}
a_0&\left(1-\frac{p(p+1)x^2}{2!}+\frac{p(p-2)(p+1)(p+3)x^4}{4!} - ...\right)+\notag\\
a_1&\left(x-\frac{(p-1)(p+2)x^3}{2!}+\frac{(p-1)(p-3)(p+2)(p+4)x^5}{4!} - ...\right)\notag
\end{align}

Self adjoint: $D((1-x^2)P_k')+k(k+1)P_k$

Rodrigues' formula: $P_n(x)=\frac{1}{2^nn!}D^n(x^2-1)^n$

Recursion: $(n+1)P_{n+1}-x(2n+1)P_n+nP_{n-1}=0$

Generating func: $\sum\limits_{n=0}^\infty t^nP_n(x) = \frac1{\sqrt{1-2xt+t^2}}$

Laplace's formula: $\frac{1}{\pi}\int\limits_0^\pi (x+ \sqrt{x^2-1\cos\phi})^nd\phi$
\subsection{Gamma/etc}
$$\Gamma(p)=\int\limits_0^\infty e^{-t}t^{p-1}dt$$

$\Gamma(p+1)=p\Gamma(p); \Gamma \sim (p-1)!$

Stirling's approx: $n! \sim n^ne^{-n}\sqrt{2n\pi}$; extends to $\Gamma$\\

(F-F thm) If an ODE has a regular sing pt, then $\exists$ at least one soln of the form $(x-x_0)^p\sum\limits_{n=0}^\infty a_n(x-x_0)^n$, +ve ratio of convergence. If the roots of $\rho$ do not differ by an integer, the solutions are linearly independant. Otherwise, they may still be LD, or the recursion formula may break down, or we get a multiple.

\subsection{Bessel}
$$J_p=\sum\limits_{n=0}^\infty \frac{(-1)^n (x/2)^{2n+p}}{n!\Gamma(n+p+1)}$$
\begin{itemize}




\item $J_\frac12=\sqrt{\frac{2}{\pi x}}\sin x$, for -0.5 cos

\item $a=x^{-p}Dx^p, a^\dagger = x^pDx^{-p}$; ($D(x^pJ_p)=x^pJ_{p-1})$
\item $xJ_p' \pm pJ_p=\pm xJ_{p\mp 1}$, $2J_p'=J_{p-1}-J_{p+1}$, $2nJ_n=x(J_{n-1}+J_{n+1})$

\item Bilateral sum of $J_n$ is 1.
Schl\"{o}milch's formula: $\sum\limits_{-\infty}^{\infty}J_n(x)t^n=\exp \frac{x}2\left(t-\frac1{t}\right)$
\item $J_m(x)=\frac1\pi\int\limits_0^\pi\cos(x\sin\theta-m\theta)d\theta$
\item Has a zero every interval of $\pi$
\item Lommel's formula: For $f(r)=\sum\limits_1^\infty A_jJ_0(\zeta_jr)$, $A_j=\frac{2}{J_1^2(\zeta_j)}\int\limits_0^1rf(r)J_0(\zeta_jr)dr$ (Orthogonal on $[0,1]$ with weight $x$)
\item $J_0(x)=\int\limits_1^\infty \frac{\sin (t x)}{\sqrt{t^2-1}} dt$
\end{itemize}
\section{Sturm-Liouville}

$$\int_\Omega u\frac{\partial v}{\partial x}d\tau= -\int_\Omega u\frac{\\partial v}{\partial x} d\tau + \int_{\partial\Omega}uv\hat i\cdot dS$$
\begin{itemize}
\item Energy method: Write $F=f_2-f_1$, now substitute in initial equation and boundary eqn. Multiply DE with $\partial_t F$ and integrate over disc covered by boundary (this is energy). Prove that this is constantly 0 (or something manageable) via derivatives. The boundary term in the integral (by parts) will vanish.
\item Generalized: $y''+\lambda\rho(x)y=0$ with boundary conditions (Dirichlet BC: $y(x_0)$ given, Neumann: $y'(x_0)$). $\rho$ is continuous and positive.
\item The eigenfunctions for such a problem on $[0,l]$ are orthogonal on the same interval with weight $\rho$
\item For dirichlet, there is 1 EF per EV. For any SL problem, there is an infinite sequence of EVs tending to infinity. The EFs cover the space of all Lipschitz functions.
\item To minimize $\int y'(t)^2 dt$ ($\langle y,\hat Ay\rangle$) subject to $\int y(t)^2\rho(t) dt=1$ ($\langle y,y\rangle=1$), we solve the corresp S-L problem, and the EF corresp to smallest EV is the solution.
\item Sturm's comparison: For weights $\rho(x)>\sigma(x)$, between two zeroes of $y_\sigma (x)$ there is at least one zero of $y_\rho(x)$
\item $\int\limits_\Omega u\frac{\partial v}{\partial x} dxdy=\int\limits_\Omega uv \hat{i} dS - \int\limits_{\partial Omega} v\frac{\partial u}{\partial x} dx dy$
\end{itemize}
\section{Fourier}
\subsection*{Discrete}
\begin{itemize}
\item Dirichlet theorem: Coefficients are $a_0=\frac1{2\pi} \int_{-\pi}^\pi f(x)dx, a_n=\frac1{\pi} \int_{-\pi}^\pi f(x)cos nx dx$. At a point of discontinuity, it will converge to the mean of the two sides.
\item Riemann-Lebesgue: The fourier coefficients of a cont fn tend to zero as $n\to\infty$
\item Weierstrau\ss~ approx: If f is continuous then given any $\epsilon>0,~\exists P(x)$ (polynomial) such that $|f(x)-P(x)|< \epsilon$ for all $x$ in interval.
\item If $s_n$s are the successive partial fourier series as trig polynomials, and $P_n$ is an arbit trig polynomial of degree at least $n$, $||f-x_n||_2\leq ||f-P||_2$, where the $L^2$ norm is $||f||_2=\sqrt{\frac1{b-a}\int_a^b |f(t)|^2 dt}$
\item Parseval: For a pair of Riemann integrable functions, then inner product of two ($\frac{1}{2\pi}\int f\bar{g}$) 
is equal to sum of products of coefficients ($a_0\bar{a_0'}+\frac12\sum a_n\bar{a_n'} + b_n\bar{b_n'}$).
\item Poisson kernel: $\frac{1-r^2}{1+r^2-2r\cos(\theta-t)}$. For a 3D system $\Delta u=0$, and the value of $u$ on the boundary is $f$, $f(\mathbf y)=\int_{\partial B} \mathrm{ker}(||\mathbf x||,\alpha) f(\mathbf x) dS$, where $\alpha$ is the angle between the two vectors. Works for 2D with norms replaced by $r$, and $\alpha$ with $\theta-t$ ($t$ is the variable of integration)

\end{itemize}
\subsection*{Continuous}
$$\hat{f}(\omega)=\int_{-\infty}^\infty f(t)e^{-i\omega t}dt;\qquad f(t)=\frac1{2\pi}\int_{-\infty}^\infty \hat{f}(\omega)e^{i\omega t}d\omega$$
\renewcommand{\arraystretch}{1.5}
\begin{tabular}{@{\hspace{50pt}}>{$}c <{$} @{\hspace{20pt}}| @{\hspace{20pt}}>{$}c <{$} @{\hspace{30pt}}  @{\hspace{30pt}}>{$}c <{$} @{\hspace{20pt}}| @{\hspace{20pt}}>{$}c <{$}}
f(t-t_0) & \hat{f}(\omega)e^{-i\omega t_0} & f(at) &\frac{1}{|a|}\hat{f}\left(\frac{\omega}{a}\right)\\
(-it)^nf(t) & \hat{f}^{(n)}(\omega) & e^{-t^2} & \sqrt{\pi } e^{-\frac{\omega^2}{4}}\\
\mathrm{rect}(t) & \mathrm{sinc}(\frac{\omega}{2})
\end{tabular}
\begin{itemize}
\item If the integral over space is finite, $\hat{f(\xi)}=\int_{-\infty}^\infty f(t) e^{-it\xi} dt$
\item Schwartz space: Space for which $t^nf^{(m)}(x)$ stays bounded for all integers. Integral over space is finite, so fourier transform exists.
\item $\widehat{f'(t)}=i\xi \hat{f}(\xi)$, and $\widehat{t f(t)}=i\frac{d}{d\xi}\hat{f}(\xi)$
\item $\int\limits_{-\infty}^\infty \hat{f}(\xi)d\xi =2\pi f(0)$
\item $\lim\limits_{k\to\infty} \int\limits_{-\infty}^\infty \frac{\sin kt}{\sin t}f(t)dt=\pi\left(f(0) +2f(\pi)+ 2f(2\pi) \cdots \right)$
\item Theta fn identity: $\pi\left(1+2e^{-a^2\pi^2}+2e^{-4a^2\pi^2}+\cdots\right)\frac{\sqrt{\pi}}{a}\left(1+2e^{-1/a^2}+2e^{-4/a^2}+\cdots\right)$
\item Poisson summation: $\sum\limits_{-\infty}^\infty f(n)=\sum\limits_{-\infty}^\infty \hat{f}(2n\pi)$
\item Parseval: $\int\limits_{-\infty}^\infty f\bar{g} =\frac1{2\pi} \int\limits_{-\infty}^\infty \hat{f}\bar{\hat{g}}$
\item Convolution: $\widehat{f*g}=\hat{f}\hat{g}$
\item Heat kernel: $\frac1{\sqrt{4\pi t}}\int_{-\infty}^\infty f(s) e^{-\frac{(x-s)^2}{4t}}ds$
\end{itemize}

\section{Useful stuff}
\subsection*{Laplace transforms}
$$\mathcal L(f)=\int_0^\infty e^{-st}f(t)dt$$

\renewcommand{\arraystretch}{1.5}

\begin{tabular}{@{\hspace{50pt}}>{$}c <{$} @{\hspace{20pt}}| @{\hspace{20pt}}>{$}c <{$} @{\hspace{30pt}}  @{\hspace{30pt}}>{$}c <{$} @{\hspace{20pt}}| @{\hspace{20pt}}>{$}c <{$}}
1 \hspace{100pt}& \frac1{s} & x^n \hspace{100pt} & \frac{n!}{s^{n+1}}\\
\sin(ax),\cos(ax) & \frac{a}{s^2+a^2},\frac{s}{s^2+a^2} & s^{ax}& \frac{1}{s-a}\\
x\mathrm{cis}(ax) & \frac{s^2-a^2+2ais}{(s^2+a^2)^2} & \sinh(ax) & \frac{a}{s^2-a^2}\\
u_c(x) & \frac{e^{-cs}}{s} & e^{ct}f(x) & \mathcal L(f)(s-c)\\
f'(x) & s\mathcal L(f)-f(0) & \frac{f(x)}{x} & \int_s^\infty \mathcal L(f)\\
\int_0^x f & \frac{\mathcal L(f)}{s} & f(cx) & \frac1{c}\mathcal L(f)(\frac{s}{c})\\
f^{(n)}(x) & \multicolumn{3}{c}{$s^n\mathcal L(f) -s^{n-1}f(0)-s^{n-2}f'(0) \cdots -f^{(n-1)}(0)$}
\end{tabular}
\end{document}
